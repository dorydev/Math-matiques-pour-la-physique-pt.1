\section{Analyse}
\subsection{Equations différentielles}

\subsubsection{EDL d'odre 1:}

Soit $f$ une fonction définie sur $[a, b]$ et continue sur ce même intervalle. On appelle équation différentielle toute équation de la forme:

$$
    \frac{df(x)}{dx} + \alpha_0 f(x) = g(x)
$$
\textbf{Résolution:}

\underline{Existence des solutions}: Si $g(x)$ est continue sur I, alors l'équation possède une infinité de solutions.\\

\underline{Conditions}:  $f(x_0)=c, c\in \mathbb{R}$ (CI) ou $f(x_p) = c, c\in \mathbb{R}$ (CP), $\Rightarrow $ Si une CI ou une CP est donnée, alors l'équation différentielle formera un problème de Cauchy.\\
L'ED sera de la forme:\\
\\
$
\left\{
    \begin{array}{ll}
        \frac{df(x)}{dx} +
            \alpha_0 f(x) = g(x) \\
        f(x_0) = c
    \end{array}
\right.
$
ou
 $
\left\{
    \begin{array}{ll}
        \frac{df(x)}{dx} + \alpha_0 f(x) = g(x) \\
        f(x_p) = c
    \end{array}
\right.
$
\\

\textbf{Equation homogène:}

Considérons $\frac{df(x)}{dx} + \alpha_0 f(x) = 0$$\iff$$\frac{df(x)}{f(x) dx} = \alpha_0$\\
Soit, 
$$f(x) = e^{(-\alpha_0 x + cte)} = A e^{-\alpha_0x}, A \in \mathbb{R}$$

\textbf{Solutions particulières:}

Forme de $g(x)$:


\begin{itemize}[label=\textbullet, font=\small]
    \item $\alpha, (\alpha \in \mathbb{R})$ (1)
    \item $P_n(x) \cos (\Omega_1x) + Q_m(x)\sin(\Omega_2x), P_n, Q_m \in \mathbf{R[X]}$ (2)
    \item $e^{\alpha x}. P_n(x), \alpha\in \mathbb{R}$ (3)
\end{itemize}

Forme de $f_p(x)$:

\begin{itemize}[label=\textbullet, font=\small]
    \item $A, (A \in \mathbb{R})$ (1)
    \item $R_n(x) \cos (\Omega_1x) + S_n(x)\sin(\Omega_1x) + T_m(x) \cos (\Omega_2x) + U_m(x)\sin(\Omega_2x)$ (2)
    \item $\alpha \neq -a_0 :e^{\alpha x}. Q_n(x), \alpha = -a_0: e^{\alpha x}. x Q_n(x)$ (3)
\end{itemize}

La solution finale de l'EDL d'ordre 1 sera $f_P(x) + f_H(x)$

\subsubsection{EDL d'ordre 2:}

Une équation différentielle linéaire d'ordre 2 est de la forme:
$$
    a_2(x) \frac{d^2y}{dx^2} + a_1(x) \frac{dy}{dx} + a_0(x) y = g(x)
$$

\textbf{Equation homogène:}

Si $g(x) = 0$, l'équation devient:
$$
    a_2(x) \frac{d^2y}{dx^2} + a_1(x) \frac{dy}{dx} + a_0(x) y = 0
$$

\textbf{Résolution:}

1. \underline{Cas à coefficients constants:}

Si $a_2, a_1, a_0$ sont constants, on résout l'équation caractéristique associée:
$$
    a_2 r^2 + a_1 r + a_0 = 0
$$
Les solutions de cette équation caractéristique déterminent la forme de la solution générale de l'équation homogène.

2. \underline{Cas général:}

Pour des coefficients non constants, on utilise des méthodes comme la variation des paramètres ou la transformation de Laplace.


\textbf{Formes du second membre:}

\begin{table}[h]
\centering
\begin{tabular}{|c|c|}
\hline
Forme de $g(x)$ & Forme de $f_P(x)$ \\ \hline
$\alpha$ & $A$ \\ \hline
$P_n(x)$ & $Q_n(x)$ \\ \hline
$e^{\beta x}$ & $B e^{\beta x}$ \\ \hline
$\cos(\omega x)$ ou $\sin(\omega x)$ & $C \cos(\omega x) + D \sin(\omega x)$ \\ \hline
$e^{\beta x} P_n(x)$ & $e^{\beta x} R_n(x)$ \\ \hline
$P_n(x) \cos(\omega x) + Q_m(x) \sin(\omega x)$ & $S_n(x) \cos(\omega x) + T_m(x) \sin(\omega x)$ \\ \hline
\end{tabular}
\end{table}

La solution générale de l'équation non homogène est la somme de la solution générale de l'équation homogène et d'une solution particulière de l'équation non homogène, c'est à dire:

$$
    y(x) = y_H(x) + y_P(x)  
$$

\subsection{Fonctions à plusieurs variables et dérivées partielles}

\subsubsection{Rappels}

\begin{table}[h]
\centering
\begin{tabular}{|c|c|c|}
\hline
Fonction & Dérivée & Domaine de dérivabilité \\ \hline
$c$ (cste) & $0$ & $\mathbb{R}$ \\ \hline
$x^n$ & $nx^{n-1}$ & $\mathbb{R}$ \\ \hline
$e^x$ & $e^x$ & $\mathbb{R}$ \\ \hline
$\ln(x)$ & $\frac{1}{x}$ & $]0, +\infty[$ \\ \hline
$\sin(x)$ & $\cos(x)$ & $\mathbb{R}$ \\ \hline
$\cos(x)$ & $-\sin(x)$ & $\mathbb{R}$ \\ \hline
$\tan(x)$ & $\sec^2(x)$ & $x \neq \frac{\pi}{2} + k\pi, k \in \mathbb{Z}$ \\ \hline
$u \pm v$ & $u' \pm v'$ & $\mathbb{R}$ \\ \hline
$uv$ & $u'v + uv'$ & $\mathbb{R}$ \\ \hline
$\frac{u}{v}$ & $\frac{u'v - uv'}{v^2}$ & $v \neq 0$ \\ \hline
$u(v(x))$ & $u'(v(x)) \cdot v'(x)$ & $\mathbb{R}$ \\ \hline
\end{tabular}
\caption{Dérivées usuelles de fonctions et d'opérations sur les fonctions}
\end{table}

\subsubsection{Dérivées partielles}

Soit $f$ une fonction définie de $\mathbb{R}^n$ à $\mathbb{R}^p$.
Une fonction de plusieurs variables est une fonction qui dépend de deux ou plusieurs variables. La dérivée partielle de $f$ par rapport à $x$, notée $\frac{\partial f}{\partial x}$, mesure le taux de variation de $f$ lorsque $x$ varie et $y$ reste constant. De même, la dérivée partielle de $f$ par rapport à $y$, notée $\frac{\partial f}{\partial y}$, mesure le taux de variation de $f$ lorsque $y$ varie et $x$ reste constant.


Pour une fonction $f(x, y)$, les dérivées partielles sont définies comme suit:
$$
\frac{\partial f}{\partial x} = \lim_{\Delta x \to 0} \frac{f(x + \Delta x, y) - f(x, y)}{\Delta x}
$$
$$
\frac{\partial f}{\partial y} = \lim_{\Delta y \to 0} \frac{f(x, y + \Delta y) - f(x, y)}{\Delta y}
$$



\subsubsection{Dérivées partielles d'ordre supérieur}

Les dérivées partielles d'ordre supérieur sont obtenues en dérivant plusieurs fois par rapport à différentes variables. Par exemple, pour une fonction $f(x, y)$, les dérivées partielles d'ordre supérieur incluent:

$$
\frac{\partial^2 f}{\partial x^2} = \frac{\partial}{\partial x} \left( \frac{\partial f}{\partial x} \right)
$$

$$
\frac{\partial^2 f}{\partial y^2} = \frac{\partial}{\partial y} \left( \frac{\partial f}{\partial y} \right)
$$

$$
\frac{\partial^2 f}{\partial x \partial y} = \frac{\partial}{\partial x} \left( \frac{\partial f}{\partial y} \right)
$$

$$
\frac{\partial^2 f}{\partial y \partial x} = \frac{\partial}{\partial y} \left( \frac{\partial f}{\partial x} \right)
$$

En général, pour une fonction $f(x_1, x_2, \ldots, x_n)$, la dérivée partielle d'ordre $k$ par rapport aux variables $x_{i_1}, x_{i_2}, \ldots, x_{i_k}$ est notée:

$$
\frac{\partial^k f}{\partial x_{i_1} \partial x_{i_2} \ldots \partial x_{i_k}}
$$


\subsection{Primitives et intégrations}

Soit $f$ une fonction définie, continue et dérivable sur $[a,b]$. Alors $F$ est une primitive de $f$ ssi:
$$
    F(x) =  \int_{a}^{b} f(x) \,dx 
$$


\subsubsection{Intégrales multiples}

Les intégrales multiples permettent de généraliser la notion d'intégrale à des fonctions de plusieurs variables. Elles sont utilisées pour calculer des volumes, des aires, des masses, etc.

\subsubsection{Intégrale double}

Soit $f(x, y)$ une fonction continue sur un domaine $D \subset \mathbb{R}^2$. L'intégrale double de $f$ sur $D$ est définie par:
$$
\iint_D f(x, y) \, dA
$$
où $dA$ représente un élément de surface infinitésimal.

\textbf{Théorème de Fubini:}

Si $f$ est continue sur un rectangle $R = [a, b] \times [c, d]$, alors:
$$
\iint_R f(x, y) \, dA = \int_a^b \left( \int_c^d f(x, y) \, dy \right) dx = \int_c^d \left( \int_a^b f(x, y) \, dx \right) dy
$$

\subsubsection{Intégrale triple}

Soit $f(x, y, z)$ une fonction continue sur un domaine $D \subset \mathbb{R}^3$. L'intégrale triple de $f$ sur $D$ est définie par:
$$
\iiint_D f(x, y, z) \, dV
$$
où $dV$ représente un élément de volume infinitésimal.

\textbf{Théorème de Fubini (extension):}

Si $f$ est continue sur un parallélépipède $R = [a, b] \times [c, d] \times [e, f]$, alors:
$$
\iiint_R f(x, y, z) \, dV = \int_a^b \int_c^d \int_e^f f(x, y, z) \, dz \, dy \, dx
$$

\subsubsection{Changement de variables}

Pour simplifier le calcul des intégrales multiples, on peut utiliser un changement de variables. Soit $u = g(x, y)$ et $v = h(x, y)$ un changement de variables bijectif et différentiable avec un jacobien non nul. Alors:
$$
\iint_D f(x, y) \, dA = \iint_{D'} f(g^{-1}(u, v)) \left| \frac{\partial(x, y)}{\partial(u, v)} \right| \, du \, dv
$$
où $\left| \frac{\partial(x, y)}{\partial(u, v)} \right|$ est le déterminant du jacobien du changement de variables.

\subsubsection{Intégrales de Riemann}

Les intégrales de Riemann généralisent la notion d'intégrale à des fonctions de plusieurs variables. Soit $f$ une fonction définie sur un domaine $D \subset \mathbb{R}^n$. L'intégrale de Riemann de $f$ sur $D$ est définie comme la limite de la somme de Riemann:
$$
\int_D f(\mathbf{x}) \, d\mathbf{x} = \lim_{\|P\| \to 0} \sum_{i=1}^m f(\mathbf{x}_i^*) \Delta V_i
$$
où $P$ est une partition de $D$, $\mathbf{x}_i^*$ est un point dans le $i$-ième sous-domaine, et $\Delta V_i$ est le volume du $i$-ième sous-domaine.

\textbf{Conditions de Riemann:}

Pour que l'intégrale de Riemann existe, il faut que $f$ soit bornée et que la somme des volumes des sous-domaines tende vers zéro lorsque la norme de la partition tend vers zéro.





\subsection{Opérateur Nabla}

L'opérateur nabla, noté $\nabla$, est un opérateur différentiel vectoriel utilisé en calcul vectoriel. Il est défini comme suit en coordonnées cartésiennes $(x, y, z)$:

$$
\nabla = \left( \frac{\partial}{\partial x}, \frac{\partial}{\partial y}, \frac{\partial}{\partial z} \right)
$$

L'opérateur nabla est utilisé pour définir plusieurs opérations importantes en analyse vectorielle, notamment le gradient, la divergence et le rotationnel.

\textbf{Gradient:}

Le gradient d'une fonction scalaire $f(x, y, z)$ est un vecteur noté $\nabla f$ et défini par:
$$
\nabla f = \left( \frac{\partial f}{\partial x}, \frac{\partial f}{\partial y}, \frac{\partial f}{\partial z} \right)
$$
Le gradient indique la direction de la plus grande variation de la fonction et sa norme représente le taux de variation maximal.

\textbf{Divergence:}

La divergence d'un champ vectoriel $\mathbf{F} = (F_1, F_2, F_3)$ est un scalaire noté $\nabla \cdot \mathbf{F}$ et défini par:
$$
\nabla \cdot \mathbf{F} = \frac{\partial F_1}{\partial x} + \frac{\partial F_2}{\partial y} + \frac{\partial F_3}{\partial z}
$$
La divergence mesure le taux de variation du flux sortant d'un point dans un champ vectoriel.

\textbf{Rotationnel:}

Le rotationnel d'un champ vectoriel $\mathbf{F} = (F_1, F_2, F_3)$ est un vecteur noté $\nabla \times \mathbf{F}$ et défini par:
$$
\nabla \times \mathbf{F} = \left( \frac{\partial F_3}{\partial y} - \frac{\partial F_2}{\partial z}, \frac{\partial F_1}{\partial z} - \frac{\partial F_3}{\partial x}, \frac{\partial F_2}{\partial x} - \frac{\partial F_1}{\partial y} \right)
$$
Le rotationnel mesure la tendance d'un champ vectoriel à tourner autour d'un point.

\subsubsection{Gradient, divergence et rotationnel en coordonnées cylindriques et sphériques}


\subsubsection{Annexes:}
\textbf{Théorème de Green:}

Le théorème de Green relie une intégrale curviligne autour d'une courbe fermée à une intégrale double sur la région délimitée par cette courbe.
\\
Soit $C$ une courbe fermée simple et $D$ la région délimitée par $C$. Si $P$ et $Q$ sont des fonctions ayant des dérivées partielles continues sur une région qui contient $D$, alors:

$$
\oint_C (P \, dx + Q \, dy) = \iint_D \left( \frac{\partial Q}{\partial x} - \frac{\partial P}{\partial y} \right) \, dA
$$

En d'autres termes, l'intégrale curviligne de $P \, dx + Q \, dy$ autour de $C$ est égale à l'intégrale double de $\frac{\partial Q}{\partial x} - \frac{\partial P}{\partial y}$ sur $D$.

\textbf{Applications du théorème de Green:}

1. Calcul de l'aire d'une région plane:
$$
A = \iint_D \, dA = \frac{1}{2} \oint_C (x \, dy - y \, dx)
$$

2. Calcul du flux d'un champ vectoriel à travers une courbe fermée:
$$
\oint_C \mathbf{F} \cdot d\mathbf{r} = \iint_D (\nabla \times \mathbf{F}) \cdot \mathbf{k} \, dA
$$
où $\mathbf{F} = (P, Q)$ et $\mathbf{k}$ est le vecteur unitaire normal à la surface.

\subsection{Développements limités et formule de Taylor}

\subsubsection{Formule de Taylor-Lagrange}
\textbf{Théorème}: Soit $f$ une fonction de classe $C^n$ sur $[a, b]$, $n+1$ fois dérivable sur $]a, b[$.
Alors, $\exists c \in ]a, b[$ tq:

$$
    f(b) = f(a) + (b-a)f'(a) + \frac{(b-a)^2)}{2!}f''(a) + ... + \frac{(b-a)^n)}{n!}f^{(n)}(a) + \frac{(b-a)^{n+1}}{(n+1)!} f^{n+1}(c)
$$

\underline{Remarque}: pour $n=0$, apparaît le TAF (théorème des accroissements finis).
\\

\textbf{Formule de Mac Laurin-Lagrange:} Si $f$ est $C^n$ sur $]-\alpha, \alpha[$ avec $f^n$ dérivable sur $]-\alpha, \alpha[ - \{0\}$, alors $\forall x \in ]-\alpha, \alpha[, \exists \theta \in  ]0, 1[$ tq:

$$
    f(0) = f(0) + xf'(0) + \frac{x^2}{2!}f''(0) + ... + \frac{x^n}{n!}f^{(n)}(0) + \frac{x^{n+1}}{(n+1)!} f^{n+1}(\theta x)
$$

\textbf{Inégalités de Taylor-Lagrange:} En ajoutant $|f^{n+1}|$ majorée par $M$ sur $]a,b[$ on obtient:
$$
    | f(b) - ( f(a) + (b-a)f'(a) + \frac{(b-a)^2)}{2!}f''(a) + ... + \frac{(b-a)^n)}{n!}f^{(n)}(a) ) | \leq M\frac{(b-a)^{n+1}}{(n+1)!}
$$

\subsubsection{Formule de Taylor-Young:} 
\textbf{Théorème:} Soit $f$ une fonction de classe $C^n$ sur $]x_0 - \alpha, x_0 + \alpha$ tq $\exists f^{(n)}(x_0)$.
Alors pour $x \in ]x_0 - \alpha, x_0 + \alpha$:
$$f(x) = f(x_0) + (x-x_0)f'(x_0) + \frac{(x-x_0)^2)}{2!}f''(x_0) + ... + \frac{(x-x_0)^n)}{n!}f^{(n)}(x_0) \varepsilon (x)$$ avec $\lim\limits_{x \to x_0} \varepsilon (x) = 0$

\textbf{Développements limités usuels:}

$e^x = 1 + x + \frac{x^2}{2!} + \frac{x^3}{3!} + ... + \frac{x^n}{n!} + x^n \varepsilon (x) = \sum_{k=0} ^{n} \frac{x^k}{k!} + x^n\varepsilon (x)$\\
$\cos(x) = 1 - \frac{x^2}{2!} + \frac{x^4}{4!} + ... + (-1)^n\frac{x^{2n}}{2n!} + x^{2n} \varepsilon (x)$\\
$\sin(x) =  x - \frac{x^3}{3!} + \frac{x^5}{5!} + ... + (-1)^n\frac{x^{2n+1}}{2n+1!} + x^{2n+1} \varepsilon (x)$\\
$\cosh(x) = 1 + \frac{x^2}{2!} + \frac{x^4}{4!} + ... + \frac{x^{2n}}{2n!} + x^{2n} \varepsilon (x)$\\
$\sinh(x) = 1 + \frac{x^3}{3!} + \frac{x^5}{5!} + ... + \frac{x^{2n+1}}{2n+1!} + x^{2n+1} \varepsilon (x)$\\
$\ln(1+x) = x -\frac{x^2}{2} + ... + (-1)^{n-1}\frac{x^n}{n} + x^n \varepsilon (x)$\\
$\ln(1-x) = - x -\frac{x^2}{2} + ... + -\frac{x^n}{n} + x^n \varepsilon (x)$\\
$\frac{1}{1-x} = 1 + x + x^2 + ... + x^n + x^n \varepsilon (x)$\\
$\frac{1}{1+x} = 1 - x + x^2 + ... +(-1)^n x^n + x^n \varepsilon (x)$\\